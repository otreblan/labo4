\documentclass[10pt, twoside]{article}
\usepackage{main}

% Aquí empieza el documento{{{
\begin{document}

%\maketitle
\thispagestyle{fancy}

\begin{center}
	LABORATORIO DE FÍSICA 2\\
	Informe del Laboratorio N° 1
\end{center}

\noindent
\begin{tikzpicture}
	\tikzmath
	{
		coordinate \xMa;
		\xMa = (\textwidth-1.6pt,0);
	}
	\draw [ultra thick](0,0) rectangle ++(\xMax*0.8,-5)
		++(0.25,0)
		rectangle (\xMax,0);

	\draw(2,-.5) node {Título del laboratorio};
	\draw({(\xMax*0.8+\xMax)/2},-.5) node {NOTA};
\end{tikzpicture}

\bigskip
\bigskip
\bigskip
\noindent
\begin{tikzpicture}
	\tikzmath
	{
		coordinate \xMa;
		\xMa = (\textwidth-0.4pt,0);
	}
	\foreach \x in {0, ..., 5}
	{
		\draw ({\x*\xMax/6},0) rectangle ++(\xMax/6,0.75);
	}
	\draw (0.62,0.4) node {\textbf{Fecha}}
		++(\xMax/3,0) node {\textbf{Hora}}
		++(\xMax/3,0)++(0.4,0) node {\textbf{Ambiente}}
		;
\end{tikzpicture}

\bigskip
\bigskip
\noindent
\begin{tikzpicture}
	\tikzmath
	{
		coordinate \xMa;
		\xMa = (\textwidth-0.4pt,0);
	}
	\foreach \y in {0,1}
	{
		\draw (0,\y) rectangle +(\xMax*0.5,-1)
			++(\xMax*0.5,0) rectangle ++(\xMax*0.25,-1)
			+(0,1) rectangle (\xMax,\y-1);
	}
	\draw (\xMax*0.25,0.7) node {\textbf{Integrantes}}
		++(\xMax*0.38,0) node {\textbf{Código}}
		++(\xMax*0.25,0) node {\textbf{Participación}}
		++(0,-0.4) node {\textbf{(\%)}}
		;
	\draw (\xMax*0.25,0.7-1) node {\textbf{Alberto Oporto Ames}}
		++(\xMax*0.38,0) node {$\mathbf{201810518}$}
		++(\xMax*0.25,0) node {\textbf{100\%}}
		;
\end{tikzpicture}

\vfill
\section{OBJETIVOS}
\section{EXPERIENCIA A}

\subsection{PROCEDIMIENTO Y ANÁLISIS}%
\subsection{ERRORES}%
\subsection{COMENTARIOS Y OBSERVACIONES}%

\section{EXPERIENCIA B}

\subsection{PROCEDIMIENTO Y ANÁLISIS}%
\subsection{ERRORES}%
\subsection{COMENTARIOS Y OBSERVACIONES}%

\section{EXPERIENCIA C}

\subsection{PROCEDIMIENTO Y ANÁLISIS}%
\subsection{ERRORES}%
\subsection{COMENTARIOS Y OBSERVACIONES}%


\end{document}
%}}}
